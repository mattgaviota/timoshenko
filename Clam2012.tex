%% LyX 2.0.0 created this file.  For more info, see http://www.lyx.org/.
%% Do not edit unless you really know what you are doing.
\documentclass[11pt,spanish]{article}
\usepackage[T1]{fontenc}
\usepackage[latin1]{inputenc}
\usepackage{amsmath}
\usepackage{amssymb}

\makeatletter
%%%%%%%%%%%%%%%%%%%%%%%%%%%%%% User specified LaTeX commands.


\usepackage{latexsym}

\topmargin 0mm
\oddsidemargin 5mm
\evensidemargin 5mm
\textwidth 150mm
\textheight 662.801 pt



\makeatother

\usepackage{babel}
\addto\shorthandsspanish{\spanishdeactivate{~<>}}

\begin{document}
\begin{center}
\textsf{~\\[14pt] %%%%%% Insertar el t�tulo de la contribuci�n %%%%%%%
Soluciones a problemas de vibraci�n de elementos estructurales con
restricciones el�sticas}
\par\end{center}

%%%%%% Nombres de los autores y sus respectivas filiaciones: %%%%%%


{\footnotesize { }{\footnotesize \par}

\begin{center}
{\footnotesize Mat�as Novoa$^{a}$, Nilsa Sarmiento$^{b}$ y Antonio
S�ngari$^{c}$\\[14pt]}
\par\end{center}{\footnotesize \par}

\begin{center}
{\footnotesize $^{a}$Universidad Nacional de Salta \\[3mm]}
\par\end{center}{\footnotesize \par}

\begin{center}
{\footnotesize $^{b}$Universidad Nacional de Salta \\[3mm]}
\par\end{center}{\footnotesize \par}

\begin{center}
{\footnotesize $^{c}$Universidad Nacional de Salta \\[3mm] }
\par\end{center}{\footnotesize \par}

{\footnotesize }}{\footnotesize \par}

%%%%%%%  Insertar el resumen %%%%%%
 %%%%%%%%%  Por favor, no usar s�mbolos o fuentes no estandard.
Al plantear el problema de contorno y autovalores que describe el
comportamiento din�mico de vigas y placas con discontinuidades en
puntos intermedios, surge que el mismo carece de soluci�n cl�sica,
debi�ndose determinar una soluci�n d�bil. La presencia de r�tulas
y de restricciones el�sticas en puntos intermedios genera condiciones
cuyas expresiones anal�ticas son totalmente an�logas a las de las
condiciones de contorno. En este trabajo se muestra la aplicaci�n
de las t�cnicas del c�lculo de variaciones, a los problemas de contorno
y autovalores que describen el comportamiento din�mico de las vigas
y placas mencionadas. Adem�s se presentan valores de los coeficientes
de frecuencias de vibraci�n y se muestran las formas modales respectivas
mediante el tratamiento moderno de los m�todos de Ritz-Galerkin y
Multiplicadores de Lagrange para obtener los autovalores aproximados.
Por otra parte, se dan detalles de la implementaci�n de un programa
desarrollado con software libre (Python) para la resoluci�n de dichos
c�lculos, esto buscando una forma r�pida, de bajo coste y completamente
amoldada a las necesidades del problema.

%%%%%%%%%%%%%%%%%%%%%%%%%%%%%%%%
%   Bibliograf�a
%%%%%%%%%%%%%%%%%%%%%%%%%%%%%%%%




{\small %%%%% Ejemplo de formato:
}{\small \par}
\begin{thebibliography}{Referencias}
{\small \bibitem{QG} V Quintana y R Grossi, Eigenfrequencies of generally
restrained Timoshenko beams, Journal of Multi-body Dynamics.\ }\textbf{\small 224(1)}{\small{}
(2010), 117-125.}\end{thebibliography}

\end{document}
