\documentclass[11pt,spanish]{article}

\usepackage{latexsym}
\usepackage{amssymb,amsmath}
\usepackage[latin1]{inputenc}
\usepackage[spanish]{babel}

\topmargin 0mm
\oddsidemargin 5mm
\evensidemargin 5mm
\textwidth 150mm
\textheight 662.801 pt

\begin{document}

\begin{center}
{\sf ~\\[14pt]
 %%%%%% Insertar el t�tulo de la contribuci�n %%%%%%%
Soluciones a problemas de vibraci�n de elementos estructurales con
restricciones el�sticas}
 \end{center}

 %%%%%% Nombres de los autores y sus respectivas filiaciones: %%%%%%

 \footnotesize{
   \begin{center}
      Mat�as Novoa$^a$,  Nilsa Sarmiento$^b$ y Antonio
S�ngari$^c$\\[14pt]

     $^a$Universidad Nacional de Salta \\[3mm]

     $^b$Universidad Nacional de Salta \\[3mm]

     $^c$Universidad Nacional de Salta\\[3mm]
   \end{center}
   }



 \normalsize
 \noindent
 %%%%%%%  Insertar el resumen %%%%%%
 %%%%%%%%%  Por favor, no usar s�mbolos o fuentes no estandard.
Al plantear el problema de contorno y autovalores que describe el
comportamiento din�mico de vigas y placas con discontinuidades en
puntos intermedios, surge que el mismo carece de soluci�n cl�sica,
debi�ndose determinar una soluci�n d�bil. La presencia de r�tulas
y de restricciones el�sticas en puntos intermedios genera condiciones
cuyas expresiones anal�ticas son totalmente an�logas a las de las
condiciones de contorno. En este trabajo se muestra la aplicaci�n
de las t�cnicas del c�lculo de variaciones, a los problemas de contorno
y autovalores que describen el comportamiento din�mico de las vigas
y placas mencionadas. Adem�s se presentan valores de los coeficientes
de frecuencias de vibraci�n y se muestran las formas modales respectivas
mediante el tratamiento moderno de los m�todos de Ritz-Galerkin y
Multiplicadores de Lagrange para obtener los autovalores aproximados.
Por otra parte, se dan detalles de la implementaci�n de un programa
desarrollado con software libre (Python) para la resoluci�n de dichos
c�lculos, esto buscando una forma r�pida, de bajo coste y completamente
amoldada a las necesidades del problema.


%%%%%%%%%%%%%%%%%%%%%%%%%%%%%%%%
%   Bibliograf�a
%%%%%%%%%%%%%%%%%%%%%%%%%%%%%%%%

{\small \begin{thebibliography}{99}

%%%%% Ejemplo de formato:

\bibitem{QG} V Quintana y R Grossi, Eigenfrequencies of generally
restrained Timoshenko beams, Journal of Multi-body Dynamics.\ {\bf  224(1)} (2010), 117-125.

\end{thebibliography}}

\end{document}

